\documentclass[../rapport.tex]{subfiles}


\begin{document}

\subsection{Building}

	Pour pouvoir build et run le projet il faut effectueur les deux commandes au sein du package 
	suivant \textit{bqsim.src} : 
		
		\begin{enumerate}
			\item\textbf{Build : }  javac reso/examples/selectiverepeat/Demo.java
			\item\textbf{Running :} java reso/examples/selectiverepeat/Demo

		\end{enumerate}

\subsection{Lancement de l'application} 

Pour l'éxécution de l'application et les choix des paramètres liés à la simualtion, nous avons
décidés d'utiliser des entrées claviers plutôt que de demande à l'utilisateur d'ajouter
un argument à la commande passée à la console pour l'exécution. 

\medskip

C'est pourquoi lors du lancement l'utilisateur est invité à choisir le nombre de paaquets
qu'il souhaite envoyer, la longueur du lien en km, le bitrate en bit/s et pour finir 
le taux de chance lié à la perte d'un paquet et/ou ACK.

\medskip

Concernant ce taux de probabilité il varie entre 0 et 1. Au cours du testing nous avons remarqué
que le format d'entrée variait en fonction du système d'eploitation en effet, sous linux, le 
double doit être représenté comme suit (x.x) et sous macOS comme suit (0,0).

\medskip 

Une fois les paramètres entrés l'application démarre et on peut suivre l'évolution de la 
simulation via la console.

\newpage
\end{document}
