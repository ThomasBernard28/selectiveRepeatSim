\documentclass[../rapport.tex]{subfiles}

\begin{document}

Le problème majeur rencontré au cours de l'implémentation est le problème responsable de la 
présence du numéro de séquence du paquet attendu par la fenêtre de réception dans le constructeur
d'un SRPacket de type ACK. 

\medskip

En effet, à cause du double constructeur dans le SRProtocol nous nous retrouvons avec 2 instances
possédant des attributs différents. Le recvBase n'était en effet mis à jour que du côté du
\textit{Receiver} ce qui posait problème lors de la réception des ACK car nous vérifions
à chaque réception d'un ACK si le recvBase est égal au nombre de paquets total afin d'écrire
le message de fin et de sauvegarder les données de la window dans un fichier. Or la réception
d'ACK se fait dans l'instance du \textit{Sender} c'est donc la raison pour laquelle lors de la 
création de l'ACK la valeur actuelle de recvBase est passée en paramètre afin de mettre celle
du \textit{Sender} à jour. 

\medskip

Un moyen de solutionner ce problème serait d'écrire directement les modifications dans le 
fichier au fur et à mesure qu'elles ont lieu. Cela aurait impliqué une trop grande révision du
code et aurait rajouté une lenteur au cours de la simulation plutôt qu'elle ait lieu une fois la 
simulation terminée. C'est pourquoi nous avons décidé de laisser le code en l'état. 

\end{document}
